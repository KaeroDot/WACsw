\documentclass[a4paper, 12pt]{article}
\usepackage{amsmath}
\usepackage{amssymb}
\usepackage{xcolor}

\title{Frequency response measurement}
\author{WAC consortium}
\begin{document}
\maketitle

\section*{Equations}
Definitions:
\begin{itemize}
	\item Digitizer: measures amplitude $D(f)$, have got frequency dependent gain $G(f)$. Time dependence of the gain is neglected.
	\item Thermal transfer standard: $U$ is output DC voltage measured by the multimeter, and the overall gain is $\Delta(f)$. Thermal transfer standard and the multimeter are supposed to be stable in time.
	\item AC source: generates signal of amplitude with non-negligible time and frequency dependence $Z(f,t)$.
\end{itemize}

One measurement of the AC source by digitizer and by the thermal transfer
standard is expressed as:
\begin{align}
	D(f, t) & = Z(f, t) \cdot G(f)      \\
	U(f, t) & = Z(f, t) \cdot \Delta(f)
\end{align}

If the measurement by digitizer and by the themral transfer standard is done simultaneously, or almost simultaneously, then:
\begin{align}
	D(f) & = Z(f) \cdot G(f)      \\
	U(f) & = Z(f) \cdot \Delta(f)
\end{align}


Two measurements are done, for measured and reference frequencies marked as
$f_\text{M}$ and $f_\text{R}$. If these measurements are done right after
another, then the time dependence of the AC source can be neglected and only
frequency dependence is taken into account:
\begin{align}
	t_\text{R} \approx t_\text{M} \implies Z\big|_{t_\text{R}} = Z\big|_{t_\text{M}}
\end{align}

Then we can write both measurements as:
\begin{align}
	D(f_\text{M}) & = Z(f_\text{M}) \cdot      G(f_\text{M}) \\
	U(f_\text{M}) & = Z(f_\text{M}) \cdot \Delta(f_\text{M}) \\
	D(f_\text{R}) & = Z(f_\text{R}) \cdot      G(f_\text{R}) \\
	U(f_\text{R}) & = Z(f_\text{R}) \cdot \Delta(f_\text{R})
\end{align}

% Measurements:
%
% \begin{align}
% 	G_H & = \frac{D_H}{Z_H}, \quad G_L = \frac{D_L}{Z_L}           \\
% 	Z_H & = \frac{U_H}{\Delta_H}, \quad Z_L = \frac{U_L}{\Delta_L}
% \end{align}
%
Combining the equations:

% \begin{align}
% 	\frac{G_{f_\text{M}}}{G_{f_\text{R}}} = \frac{D_{f_\text{M}} \cdot \Delta_{f_\text{M}}}{U_{f_\text{M}}} \cdot \frac{U_{f_\text{R}}}{D_{f_\text{R}} \cdot \Delta_{f_\text{R}}}
% \end{align}
\begin{align}
	\frac{G(f_\text{M})}{G(f_\text{R})} = \frac{D(f_\text{M})}{D(f_\text{R})} \cdot \frac{U(f_\text{R})}{U(f_\text{M})} \cdot \frac{\Delta(f_\text{M})}{\Delta(f_\text{R})}
\end{align}


For $f_\text{M}$ varied through whole input bandwidth of the digitizer, the
ratio returns the frequency response of the digitizer related to the gain at
reference frequency $f_\text{R}$. If the reference frequency is small enough,
the gain $G_{f_\text{R}}$ can be measured using DC method.

\textcolor{red}{\text{DC offset is not result of measurement and processing!}}

AC-DC measurements:

\begin{align}
	\Delta_L        & = 1 + \delta(f)                                    \\
	U_{\rm AC_{in}} & = U_{\rm DC_{in}} \cdot (1 + \delta \cdot 10^{-1})
\end{align}

With:

\begin{align}
	\delta_{792} = \frac{U_{\rm out_{AC}} - U_{\rm out_{DC}}}{n\cdot U_{\rm out_{DC}}}
\end{align}

\end{document}
